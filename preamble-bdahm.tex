\usepackage[utf8]{inputenc}
\usepackage{amsmath,amssymb}
\usepackage{tikz}
\usepackage{csquotes}
\usepackage{booktabs}
\usepackage{xspace}
\usepackage[shortlabels]{enumitem}

\usepackage{bm}

\usepackage{pifont}
\usepackage{url}

\usepackage[algo2e,ruled,vlined,linesnumbered, noend]{algorithm2e}
\SetCommentSty{textit}
\DontPrintSemicolon
\IncMargin{-\parindent}
\SetAlCapHSkip{0pt}
\SetAlgoLined
\makeatletter
\newcommand{\removelatexerror}{\let\@latex@error\@gobble}
\makeatother
\SetKwProg{Function}{Function}{}{}
\SetKw{Continue}{continue}
\SetKwFunction{Activate}{Activate}
\SetKwFunction{rate}{rate}
\SetKwFunction{coarsen}{coarsen}
\SetKwFunction{DeltaGainUpdate}{DeltaGainUpdate}
\SetKwFunction{UpdateGainCache}{UpdateGainCache}
\SetKwFunction{AdaptiveHashTableConstruction}{AdaptiveHashTableConstruction}
\SetKwFunction{BFSTraversal}{BFSTraversal}
\DeclareMathOperator*{\argmax}{\arg\!\max}

\newcommand{\cmark}{{\color{KITgreen}\ding{51}}}%
\newcommand{\xmark}{{\color{KITred}\ding{55}}}%
\newcommand{\itemcolor}{KITblue}



\newcommand{\quickcite}[1]{\textcolor{KITblue}{\scriptsize{[#1]}}}
\newcommand{\imgsource}[1]{\textcolor{KITblue}{\tiny [#1]}}
\newcommand{\header}[1]{{\bf \footnotesize \color{KITgreen} #1}}
\newcommand{\twoparttitle}[2]{{\footnotesize \bf \color{KITgreen} #1 $|$} #2}

\newcommand{\highlight}[1]{\textcolor{KITblue}{#1}}
\newcommand{\alert}[1]{\textcolor{KITred}{#1}}

\usepackage{graphicx}

\definecolor{grau}{rgb}{.5,0.5,0.5}
\definecolor{darkred}{rgb}{.7,0.05,0.2}
%\definecolor{darkyellow}{rgb}{0.9,0.8,0}
%\definecolor{darkgreen}{rgb}{0,0.6,0}
%\definecolor{mymagenta}{rgb}{0.5,0.5,0}
\definecolor{mycyan}{rgb}{0,0.6,0.6}
\definecolor{myred}{rgb}{1,0.2,0.2}
\definecolor{mygreen}{rgb}{0,.5,0}
%\definecolor{KITred}{rgb}{0.60,0,0.32}
%\definecolor{KITred70}{rgb}{0.88,0,0.48}
\definecolor{grey}{rgb}{.3,0.3,0.3}
\definecolor{darkred}{rgb}{.7,0.05,0.2}
\definecolor{darkyellow}{rgb}{0.9,0.8,0}
\definecolor{darkgreen}{rgb}{0,0.5,0}
\definecolor{mymagenta}{rgb}{0.5,0.5,0}
\definecolor{white}{rgb}{1,1,1}

\newcommand{\rot}[1]{{\color{myred}#1}}
\newcommand{\grau}[1]{{\color{grey}#1}}
\newcommand{\cyan}[1]{{\color{mycyan}#1}}
\newcommand{\blau}[1]{{\color{KITblue}#1}}
\newcommand{\magenta}[1]{{\color{mymagenta}#1}}
%\newcommand{\gruen}[1]{{\color{darkgreen}#1}}
\newcommand{\gruen}[1]{{\color{mygreen}#1}}

% Macropackage of Peter Sanders
% =============================
%
% SE, 07.01.99: \prob, \expect, \var modified
% SE, 26.01.99: joined with macros of SE
% allgemeine mathematische Notation
%\newcommand{\Id}[1]{\ensuremath{\mathsf{#1}}}
\newcommand{\hdelta}{\rot{\hat{\delta}}}
\newcommand{\Bdelta}{\blau{\bar{\delta}}}
\newcommand{\folgt}{\longrightarrow}
\newcommand{\equivalent}{\longleftrightarrow}
\newcommand{\seqGilt}[2]{\left\langle #1\gilt #2\right\rangle}
\newcommand{\Id}[1]{\ensuremath{\text{{\sf #1}}}}
\newcommand{\ceil}[1]{\left\lceil #1\right\rceil}
\newcommand{\floor}[1]{\left\lfloor #1\right\rfloor}
\newcommand{\abs}[1]{\left| #1\right|}
\newcommand{\norm}[1]{\left\|#1\right\|}
\newcommand{\enorm}[1]{\norm{#1}_{2}}
\newcommand{\sumnorm}[1]{\norm{#1}_{1}}
\newcommand{\maxnorm}[1]{\norm{#1}_{\infty}}
\newcommand{\xor}{\oplus}
\newcommand{\set}[1]{\left\{ #1\right\}}
\newcommand{\gilt}{:}
%\newcommand{\sodass}{\,:\,}
\newcommand{\sodass}{\mid}
\newcommand{\setGilt}[2]{\left\{ #1\gilt #2\right\}}
\newcommand{\Def}{:=}
\newcommand{\zvektor}[2]{\left(#1,#2\right)}
\newcommand{\vek}[1]{{\bf#1}}
\newcommand{\vektor}[2]{\left(\begin{smallmatrix}#1\\#2\end{smallmatrix}\right)}
\newcommand{\vektord}[3]{\left(\begin{smallmatrix}#1\\#2\\#3\end{smallmatrix}\right)}
\newcommand{\condition}[1]{\left[#1\right]}
\newcommand{\binomial}[2]{\binom{#1}{#2}}
\newcommand{\even}{\mathrm{even}}
\newcommand{\odd}{\mathrm{odd}}
\newcommand{\mymod}{\,\bmod\,}
% \newcommand{\divides}{|}


% Typen
\newcommand{\nat}{\mathbb{N}}
\newcommand{\natnull}{\mathbb{N}_{0}}
%\newcommand{\natless}[1]{\mathbb{N}_{&lt;#1}}
\newcommand{\natless}[1]{\mathbb{N}_{#1}}
\newcommand{\nplus}{\mathbb{N}_+}
\newcommand{\real}{\mathbb{R}}
\newcommand{\rplus}{\mathbb{R}_+}
\newcommand{\rnneg}{\mathbb{R}_*}
\newcommand{\integer}{\mathbb{Z}}
% \newcommand{\intint}[2]{\set{#1,\ldots, #2}}
\newcommand{\intint}[2]{{#1}..{#2}}
\newcommand{\realrange}[2]{\left[#1, #2\right]}
\newcommand{\realrangeo}[2]{\left(#1, #2\right)}
\newcommand{\realrangelo}[2]{\left(#1, #2\right]}
\newcommand{\realrangero}[2]{\left[#1, #2\right)}
\newcommand{\unitrange}[2]{\realrange{0}{1}}
\newcommand{\bool}{\set{0,1}}
%\newcommand{\boolean}{\mathbb{B}}
%\newcommand{\mapping}[2]{#1\rightarrow #2}
\newcommand{\mapping}[2]{{#2}^{#1}}
\newcommand{\powerset}[1]{{\cal P}\left(#1\right)}
\newcommand{\NP}{\ensuremath{\mathbf{NP}}\xspace}
\newcommand{\Bild}{\mathbf{Bild}\:}

% Typannotation
\newcommand{\withtype}[1]{\in#1}

% Wahrscheinlichkeitsrechnung
\newcommand{\prob}[1]{{\mathbb{P}}\left[#1\right]}
\newcommand{\condprob}[2]{{\mathbb{P}}\left(#1\;|\;#2\right)}
\newcommand{\condexpect}[2]{{\mathbb{E}}\left(#1\;|\;#2\right)}
\newcommand{\expect}[1]{{\mathbb{E}}\left[#1\right]}
\newcommand{\var}{{\mathbb{V}}}
\newcommand{\quant}[2]{\tilde{#1}_{#2}}

% asymptotische Notation
\newcommand{\whpO}[1]{\tilde{\mathrm{O}}\left( #1\right)}
\newcommand{\Oschlange}[1]{\tilde{\mathrm{O}}\!\left( #1\right)}
\newcommand{\Ohh}[1]{\mathrm{O}\!\left( #1\right)}
\newcommand{\Oh}[1]{\mathrm{O}\!\left( #1\right)}
\newcommand{\Ohlarge}[1]{\mathrm{O}\!\left( #1\right)}
% \newcommand{\Oh}[1]{{\mathrm{O}}(#1)}
\newcommand{\Ohsmall}[1]{\mathrm{O}(#1)}
\newcommand{\oh}[1]{\mathrm{o}\!\left( #1\right)}
\newcommand{\Th}[1]{\Theta\!\left( #1\right)}
\newcommand{\Om}[1]{\Omega\left(#1\right)}
\newcommand{\om}[1]{\omega\!\left( #1\right)}
\newcommand{\Oleq}{\preceq}

% local reference
\newcommand{\lref}[1]{\ref{\labelprefix:#1}}
\newcommand{\llabel}[1]{\label{\labelprefix:#1}}
\newcommand{\labelprefix}{} % later redefined using renewcommand

% Diskussion
\newcommand{\discussionsize}{\small}
\newenvironment{discussion}{\par\discussionsize}{\par}

% open issues
%\marginparwidth5cm
\marginparpush2mm
\marginparsep1mm 
\newcommand{\notiz}[1]{}
\newcommand{\frage}[1]{}
%\newcommand{\frage}[1]{\makebox[0cm]{$\bigotimes$}\marginpar{\tiny #1}}

\newcommand{\mysubsubsection}[1]{\vspace{2mm}\noindent{\bf #1 }}

% punkt am ende von display math
\newcommand{\punkt}{\enspace .}

% Pseudocode Unterst\"utzung
\newenvironment{code}{\noindent%\sf%
\begin{tabbing}%
\hspace{2em}\=\hspace{2em}\=\hspace{2em}\=\hspace{2em}\=\hspace{2em}\=%
\hspace{2em}\=\hspace{2em}\=\hspace{2em}\=\hspace{2em}\=\hspace{2em}\=%
\kill}{\end{tabbing}}

% 1=pos, 2=llable, 3=caption
\newcommand{\labelcommand}{}
\newcommand{\captiontext}{}
\newsavebox{\codeparam}
\newcounter{lineNumber}
\newenvironment{disscodepos}[3]{%
\renewcommand{\labelcommand}{#2}%
\renewcommand{\captiontext}{#3}%
\sbox{\codeparam}{\parbox{\textwidth}{#3}}%
\begin{figure}[#1]\begin{center}\begin{code}\setcounter{lineNumber}{1}}{%
\end{code}\end{center}\caption{\llabel{\labelcommand}\captiontext}\end{figure}}

\newenvironment{disscode}[2]{\begin{disscodepos}{htb}{#1}{#2}}%
{\end{disscodepos}}

% code in text 
%\newcommand{\codel}[1]{{\sf #1}}
%\newcommand{\codem}[1]{\mathsf{#1}}
\newcommand{\codel}[1]{\mbox{\rm "`#1"'}}
\newcommand{\codem}[1]{\mathrm{#1}}

\newcommand{\Assert}{{\bf\color{KITblue} assert\ }}
\newcommand{\Invariant}{{\bf\color{KITblue} invariant\ }}
\newcommand{\Old}[1]{{\bf\color{KITblue} old}\ensuremath{(}#1\ensuremath{)}}
\newcommand{\Class}{{\bf\color{KITblue} Class\ }}
\newcommand{\Constant}{{\bf\color{KITblue} Constant\ }}
\newcommand{\Array}{{\bf\color{KITblue} Array\ }}
\newcommand{\Of}{{\bf\color{KITblue} of\ }}
\newcommand{\Functii} {{\bf\color{KITblue} Function\ }}
\newcommand{\Funct}[3]{\Functii #1\Declare{{\rm (}{#2\rm )}}{#3}}
\newcommand{\TFunct}[2]{\Functii #1\Declare{}{#2}}
\newcommand{\ProcedP}{{\bf\color{KITblue} Procedure\ }}
\newcommand{\Operator}{{\bf\color{KITblue} Operator\ }}
\newcommand{\Type}{{\bf\color{KITblue} Type\ }}
\newcommand{\Address}{{\bf\color{KITblue} address of\ }}
\newcommand{\Pointer}{{\bf\color{KITblue} Pointer\ }}
\newcommand{\Points}{\ensuremath{\rightarrow}}
\newcommand{\Allocate}{{\bf\color{KITblue} allocate\ }}
\newcommand{\This}{{\bf\color{KITblue} this\ }}
\newcommand{\Null}{{\bf\color{KITblue} null\ }}
\newcommand{\Dispose}{{\bf\color{KITblue} dispose\ }}
\newcommand{\Deallocate}{{\bf\color{KITblue} dispose\ }}
\newcommand{\Delete}{{\bf\color{KITblue} dispose\ }}
\newcommand{\Process}{{\bf\color{KITblue} process\ }}
\newcommand{\WhileP}    {{\bf\color{KITblue} while\ }}
\newcommand{\RepeatP}   {{\bf\color{KITblue} repeat\ }}
\newcommand{\UntilP}    {{\bf\color{KITblue} until\ }}
\newcommand{\LoopP}     {{\bf\color{KITblue} loop\ }}
\newcommand{\Exit}     {{\bf\color{KITblue} exit\ }}
\newcommand{\Goto}     {{\bf\color{KITblue} goto\ }}
\newcommand{\Do}       {{\bf\color{KITblue} do\ }}
\newcommand{\Od}       {{\bf\color{KITblue} od\ }}
\newcommand{\Dopar}       {{\bf\color{KITblue} dopar\ }}
\newcommand{\ForP}      {{\bf\color{KITblue} for\ }}
\newcommand{\Foreach}      {{\bf\color{KITblue} foreach\ }}
\newcommand{\Rof}      {{\bf\color{KITblue} rof\ }}
\newcommand{\Is}{\mbox{\rm := }}
\newcommand{\Forall}      {{\bf\color{KITblue} forall\ }}
\newcommand{\ForFromTo}[3]{{\ForP $#1$ \Is $#2$ \To $#3$ \Do}}
\newcommand{\ForFromToWhile}[4]{{\ForP $#1$ \Is $#2$ \To $#3$ \WhileP $#4$ \Do}}
\newcommand{\ForFromDowntoWhile}[4]{{\ForP $#1$ \Is $#2$ \Downto $#3$ \WhileP $#4$ \Do}}
\newcommand{\ForFromDownto}[3]{{\ForP $#1$ \Is $#2$ \Downto $#3$ \Do}}
%\newcommand{\ForFromWhile}[3]{{\ForP $#1$ \Is $#2$ \To $\infty$ \WhileP $#3$ \Do}}
\newcommand{\ForFromWhile}[3]{{\ForP ($#1$ \Is $#2$;\,\, $#3$;\,\, $#1\Increment$) }}
%\newcommand{\ForFromdownWhile}[3]{{\ForP $#1$ \Is $#2$ \Downto $-\infty$ \WhileP $#3$ \Do}}
\newcommand{\ForFromdownWhile}[3]{{\ForP ($#1$ \Is $#2$;\,\, $#3$;\,\, $#1\Decrement$) }}
\newcommand{\ForFromToStep}[4]{{\ForP $#1$ \Is $#2$ \To $#3$ \Step $#4$  \Do}}
\newcommand{\ForFromDowntoStep}[4]{{\ForP $#1$ \Is $#2$ \Downto $#3$  \Step $#4$ \Do}}
\newcommand{\ForFromStepWhile}[4]{{\ForP $#1$ \Is $#2$ \To $\infty$ \Step $#3$  \WhileP $#4$ \Do}}
\newcommand{\ForFromdownStepWhile}[4]{{\ForP $#1$ \Is $#2$ \Downto $-\infty$ \Step $#3$  \WhileP $#4$ \Do}}
%\newcommand{\ForFromTo}[3]{{\ForP $#1\in #2..#3$ \Do}}
%\newcommand{\ForFromToDown}[3]{{\ForP $#1\in #2..#3$ {\bf\color{KITblue} downwards} \Do}}
%\newcommand{\ForFromWhile}[3]{{\ForP $#1\in #2..$ \WhileP $#3$ \Do}}
%\newcommand{\ForFromDownWhile}[3]{{\ForP $#1\in #2..$ {\bf\color{KITblue} downwards} \WhileP $#3$ \Do}}
%\newcommand{\ForFromToStep}[4]{{\ForP $#1\in #2..#3$ \Step $#4$ \Do}}
%\newcommand{\ForFromToDownStep}[4]{{\ForP $#1\in #2..#3$ {\bf\color{KITblue} downwards} \Step $#4$  \Do}}
%\newcommand{\ForFromWhileStep}[4]{{\ForP $#1\in #2..$ \WhileP $#3$ \Step $#4$  \Do}}
%\newcommand{\ForFromDownWhileStep}[4]{{\ForP $#1\in #2..$ {\bf\color{KITblue} downwards} \WhileP $#3$ \Step $#4$  \Do}}
\newcommand{\To}       {{\bf\color{KITblue} to\ }}
\newcommand{\Step}       {{\bf\color{KITblue} step\ }}
\newcommand{\Downto}       {{\bf\color{KITblue} downto\ }}
\newcommand{\IfP}       {{\bf\color{KITblue} if\ }}
\newcommand{\Endif}    {{\bf\color{KITblue} endif\ }}
\newcommand{\Fi}       {{\bf\color{KITblue} fi\ }}
\newcommand{\ThenP}     {{\bf\color{KITblue} then\ }}
\newcommand{\ElseP}     {{\bf\color{KITblue} else\ }}
\newcommand{\Elsif}    {{\bf\color{KITblue} else if\ }}
\newcommand{\ReturnP}   {{\bf\color{KITblue} return\ }}
\newcommand{\Set}      {{\bf\color{KITblue} set\ }}
%\newcommand{\Boolean}  {{\bf\color{KITblue} boolean\ }}
\newcommand{\Boolean}  {\ensuremath{\set{0,1}}}
\newcommand{\Integer}  {$\integer$}
%\newcommand{\True}     {{\bf\color{KITblue} true\ }}
%\newcommand{\False}    {{\bf\color{KITblue} false\ }}
\newcommand{\True}     {\mathtt{1}}
\newcommand{\False}    {\mathtt{0}}
\newcommand{\Bitand}   {{\bf\color{KITblue} bitand\ }}
%\newcommand{\Var}      {{\bf\color{KITblue} var\ }}
%\newcommand{\Xor}       {{\bf\color{KITblue}\ xor\ }}
%\newcommand{\Not}       {{\bf\color{KITblue}\ not\ }}
%\newcommand{\Or}       {{\bf\color{KITblue}\ or\ }}
\newcommand{\Xor}       {\ensuremath{\oplus}}
\newcommand{\Not}       {\ensuremath{\neg}}
\newcommand{\Or}       {\ensuremath{\vee}}
%\newcommand{\And}       {\ensuremath{\wedge}}
\newcommand{\Div}       {{\bf\ div\ }}
\newcommand{\Mod}       {{\bf\ mod\ }}
\newcommand{\Decrement}  {\ensuremath{\mathbf{-}\mathbf{-}\ }}
\newcommand{\Increment}  {\ensuremath{\mathbf{+}\mathbf{+}\ }}
\newcommand{\End}       {{\bf\color{KITblue} end\ }}
\newcommand{\Endfor}       {{\bf\color{KITblue} endfor\ }}
%\newcommand{\Rem}[1]   {{\bf\color{KITblue} (*~}{\rm#1}{\bf\color{KITblue} ~*)}}
\newcommand{\Rem}[1]   {{\bf\color{KITblue} //\hspace{0.5mm}{\rm#1}}}
% rechtsbuendiger Kommentar
%\newcommand{\RRem}[1]   {\`{$\mathbf{(*}$~ }{\rm#1}{~$\mathbf{*)}$}}
%\newcommand{\RRem}[1]   {\`{\bf\color{KITblue} --\hspace{0.5mm}--~}{\rm#1}}
\newcommand{\RRem}[1]   {\`{\bf //\hspace{0.5mm}~}{\rm#1}}
\newcommand{\Flush}[1]   {\`{\bf \hspace{0.5mm}~}{\rm#1}}
\newcommand{\RRemNL}[1]   {\`{\bf (*~ }{\rm#1}{\bf ~*)}%
{\tiny\arabic{lineNumber}}\stepcounter{lineNumber}}
\newcommand{\Declare}[2]{#1\mbox{ \rm : }#2}
\newcommand{\DeclareInit}[3]{#1$=$#3 \mbox{ \rm : }#2}
%\newcommand{\At}[1]{\left\langle#1\right\rangle}
\newcommand{\At}[1]{@#1}
\newcommand{\NL}{\`{\tiny\arabic{lineNumber}}\stepcounter{lineNumber}}

% Parallelverarbeitungspseudocode
\newcommand{\iProc}{i_\mathrm{PE}}

% Parameter 1=pos, 2=xsize, 3=filename, 4=llabel, 4=caption
\newcommand{\dissepslong}[5]{\begin{figure}[#1]\begin{center}%
\epsfxsize#2\leavevmode\epsfbox{#3.eps}%
\end{center}\caption{\llabel{#4}#5}\end{figure}}

\newcommand{\dissepspos}[4]{\dissepslong{#1}{#2}{\labelprefix/#3}{#3}{#4}}
\newcommand{\disseps}[3]{\dissepspos{htb}{#1}{#2}{#3}}

% Beweise
\newdimen\endofsize\endofsize=0.5em
\def\endofbeweis{~\quad\hglue\hsize minus\hsize
                 \hbox{\vrule height \endofsize width
\endofsize}\par}
% gibt es in amsmath schon
\newenvironment{myproof}{\begin{proof}}{\endofbeweis\end{proof}}
% \newcommand{\platsch}{\hglue\hsize minus\hsize}


% Definitions of Sebastian Egner
% ==============================



% \Z gives the nice `Z&apos; to denote the integers
%\newcommand{\Z}[0]{%
%{\rm\sf\setbox0=\hbox{Z}Z\makebox[-\wd0]{}\,Z}}
\newcommand{\Z}[0]{{\mathbb{Z}}}

% \Q gives the nice `Q&apos; to denote the rationals
%\newcommand{\Q}[0]{%
%{\rm\sf\setbox0=\hbox{Q}Q\makebox[-\wd0]{}\,I\,\,}}
%\newcommand{\Q}[0]{{\mathbb{Q}}}

% \R gives the nice `R&apos; to denote reals
%\newcommand{\R}[0]{\mbox{$\rm\sf I\!R$}}
\newcommand{\R}[0]{{\mathbb{R}}}

% \C gives the nice `C&apos; to denote complex numbers
%\newcommand{\C}[0]{\mbox{$\rm\sf I\!\!\!C$}}
\newcommand{\C}[0]{{\mathbb{C}}}

% \F gives the nice `F&apos; to denote a generic field
\newcommand{\F}[0]{{\rm\sf F}}

% \GL gives the nice `GL&apos; to denote invertible matrices
\newcommand{\GL}[0]{{\rm\sf GL}}

% \GF gives the nice &apos;F&apos; to denote finite fields
%\newcommand{\GF}[1]{\mbox{$\rm\sf I\!R_{#1}$}}
\newcommand{\GF}[1]{{\mathbb{F}_{#1}}}

% bigsqcap_{#1} (has been forgotten in TeX)
\newcommand{\bigsqcap}[1]{
  \hspace{-1.0em}\raisebox{-1.0ex}{
    \renewcommand{\arraystretch}{0.5}
    $\begin{array}{c}
      \mbox{\LARGE$\sqcap$}  \\
      \mbox{\scriptsize$#1$}
    \end{array}$}\!\!}

% combinations n choose k
\newcommand{\combinations}[2]{%
{\renewcommand{\arraystretch}{0.8}%
 \setlength{\arraycolsep}{0.2em}%
 \left( \begin{array}{c} {#1} \\ {#2} \end{array} \right)}}

% \smallexpr{&lt;expr&gt;} puts &lt;expr&gt; in \footnotesize
\newcommand{\smallexpr}[1]{\mbox{\footnotesize$#1$}}

% \smallsmallexpr{&lt;expr&gt;} puts &lt;expr&gt; in \scriptsize
\newcommand{\smallsmallexpr}[1]{\mbox{\scriptsize$#1$}}

% \begin{theorem} .. \end{theorem} gives a theorem
%\newtheorem{theorem}{Satz}

% \begin{lemma} .. \end{lemma} gives a lemma
%\newtheorem{lemma}[theorem]{Lemma}
%\newtheorem{conjecture}[theorem]{Conjecture}

% a prominent problem
%\newtheorem{problem}[theorem]{Problem}

% \begin{definition} .. \end{definition} gives a def.
%\newtheorem{definition}[theorem]{Definition}

%\newtheorem{corollary}[theorem]{Corollary}

% \begin{algorithm} .. \end{algorithm} gives an algorithm
%\newenvironment{algorithm}%
%{\par\vspace{1.0ex}\noindent{\bf Algorithm}\ }%
%{\par\vspace{-1.5ex}\noindent}

% \begin{proof} .. \end{proof} gives a proof
%\newenvironment{proof}%
%{\vspace{0.5ex}\noindent{\bf Proof:}\ }%
%{\par\vspace{0.5ex}\noindent}%

% q.e.d. sign
\newcommand{\myqed}[0]{\hfill\rule{1ex}{1ex}}
%\newcommand{\qed}[0]{~\mbox{$\triangleleft$}}
%\newcommand{\qed}[0]{\hfill\mbox{$\Box$}}


\newcommand{\bdelta}{\bar{\delta}}
%\newcommand{\tdelta}{\tilde{\delta}}
%\newcommand{\tA}{\tilde{A}}
%\newcommand{\tq}{\tilde{q}}
%\newcommand{\tF}{\tilde{F}}
\newcommand{\tdelta}{\delta&apos;}
\newcommand{\tA}{A&apos;}
\newcommand{\tq}{q&apos;}
\newcommand{\tF}{F&apos;}
\frage{auch in skript!}
\newcommand{\Mequiv}{M_{\equiv}}
\newcommand{\blank}{\ensuremath{\sqcup}}
%\newcommand{\Poly}{\text{\bf P}}
\newcommand{\Poly}{\ensuremath{\mathbf{P}}\xspace}
%\newcommand{\NP}{\mathcal{NP}}
\newcommand{\FPTAS}{\ensuremath{\mathbf{FPTAS}}\xspace}
\newcommand{\PTAS}{\ensuremath{\mathbf{PTAS}}\xspace}
\newcommand{\APX}{\ensuremath{\mathbf{APX}}\xspace}
\newcommand{\algname}{sss-sort}
\newcommand{\Algname}{Sss-sort}
\newcommand{\Univ}{\mathcal{H}}
\newcommand{\SProduct}{\mathcal{H^{\cdot}}}
\newcommand{\Hdot}{\ensuremath{H^{\cdot}}}
\newcommand{\Hbit}{\ensuremath{H^{\oplus}}}
\newcommand{\Hshift}{\ensuremath{H^{\gg}}}
\newcommand{\Hmatrix}{\ensuremath{H^{\times}}}
\newcommand{\Hmult}{\ensuremath{H^{*}}}
\newcommand{\Hfool}{\ensuremath{H^{\mathrm{fool}}}}
\newcommand{\Htable}{\ensuremath{H^{[]}}}
\newcommand{\Htableplus}{\ensuremath{H^{\oplus[]}}}
\newcommand{\Lmax}{L_{\mathrm{max}}}

\newcommand{\population}{\mathsf{pop}}
\newcommand{\neighbors}{\mathcal{N}}
\newcommand{\candidates}{\mathsf{cand}}
\newcommand{\current}{\mathsf{curr}}
%\renewcommand{\solution}{\hat{\myx}}
\newcommand{\capacity}{\mathrm{cap}}
\newcommand{\taboosize}{L}
\newcommand{\taboo}{k}
\newcommand{\sol}{\mathcal{L}}
\newcommand{\universe}{\mathcal{U}}
\newcommand{\oracle}{\mathcal{P}}
\newcommand{\fopt}{f^*}
\newcommand{\fupper}{\hat{f}}
\newcommand{\flower}{\check{f}}
\newcommand{\xlower}{\check{\myx}}
\newcommand{\myv}[1]{\mathbf{#1}}
\newcommand{\myx}{\myv{x}}
\newcommand{\myy}{\myv{y}}

%\newtheorem{theorem}{Theorem}
%\newtheorem{lemma}[theorem]{Lemma}
%\newtheorem{corollary}[theorem]{Corollary}
\newcommand{\Tprim}{T_{\mathrm{Prim}}}
\newcommand{\Tprefix}{T_{\mathrm{prefix}}}
%\newtheorem{corollary}[theorem]{Korollar}
\newcommand{\Exp}[1]{{\rm E}[ #1 ]}
%\newcommand{\Invariant}{{\bf invariant\ }}
%\newcommand{\Increment}  {\ensuremath{\mathbf{+}\mathbf{+}\ }}
%\newcommand{\Assert}{{\bf assert\ }}
%\newcommand{\Class}{{\bf Class\ }}
%\newcommand{\Array}{{\bf Array\ }}
%\newcommand{\Declare}[2]{#1\mbox{ \rm : }#2}
%\newcommand{\DeclareInit}[3]{#1$=$#3 \mbox{ \rm : }#2}
%\newcommand{\Of}{{\bf of\ }}
%\newcommand{\Funct}[3]{\Function #1\Declare{{\rm (}{#2\rm )}}{#3}}
\newcommand{\seq}[1]{\langle#1\rangle}
\newcommand{\maxL}{L_{\max}}
\newcommand{\optL}{{L_{\max}^{*}}}
\newcommand{\Dphys}{D}
\newcommand{\Batch}{b}
\newcommand{\Bdist}[2]{\mathcal{B}(#1,#2)}
\newcommand{\Ldelta}{L_{\Delta}}
\newcommand{\Pdelta}{P_{d}}
\newcommand{\absDelta}{d}
\newcommand{\calN}{{\cal N}}
\newcommand{\hash}{h}
\newcommand{\height}{H}
\newcommand{\hmax}{H_{\mathrm{max}}}
\newcommand{\myc}{\gamma}
\newcommand{\bavg}{b_{\epsilon}}
%\newcommand{\bavg}{\bar{b}}
\newcommand{\Pepsilon}{{P_d^{\epsilon}}}
% conditions: 
% epsilon &lt; 1/2
% alpha &lt; e^(-2/(1-epsilon))
\newcommand{\fastalpha}{16}
\newcommand{\fastepsilon}{1/5}
\newcommand{\ExtClassII}[6]{\mbox{\sc #1-I-O}_{#2,#3,#4}(#5,#6)}
\newcommand{\MHDMII}[5]{\ExtClassII{MHDM}{#1}{#2}{#3}{#4}{#5}}
\newcommand{\IPDMII}[5]{\ExtClassII{IPDM}{#1}{#2}{#3}{#4}{#5}}
\newcommand{\Ga}{G_{a}} % allocation graph
\newcommand{\Gs}{G_{s}} % schedule. G&apos; is worst choice
\newcommand{\Path}[2]{(#1,\ldots,#2)}
\newcommand{\DeltaBig}{\bar{\Delta}}
\newcommand{\Bdach}{\hat{B}}
%\input{mpilogo}
%\input{./allmakros}
\newcommand{\mylevels}{\protect\epsffile{levels.eps}}
\newsavebox{\logoalt}
\newdimen\semcm
\semcm=1cm
%\sbox{\logoalt}{\epsfysize1.5\semcm\epsffile{kitsmall.eps}}
\newsavebox{\lvs}
\sbox{\lvs}{}
%\markright{\color{KITgreen}{\footnotesize \hspace*{1\semcm}Sanders: Algorithmen I {\tiny\today} \hspace*{\fill}\protect\raisebox{-0.5\semcm}[1.5\semcm][1\semcm]{\usebox{\lvs}}\hspace*{\fill}\protect\raisebox{-1\semcm}[1.5\semcm][1\semcm]{\usebox{\logoalt}}\hspace*{4mm}}}
%%\sbox{\logo}{\color{blue}\mpipicture{0.3pt}{\thinlines}{{\footnotesize\INFORMATIK}}}
%%\sbox{\logo}{\epsfxsize2.5\semcm\epsffile{unilogo.ps}}
%\sbox{\logo}{\epsfysize1.5\semcm\epsffile{logo_ffi_cmyka.eps}}
%\newsavebox{\lvs}
%\sbox{\lvs}{}
%\markright{\color{blue}{\footnotesize \hspace*{1\semcm}Sanders: Algorithmen I {\tiny\today}\hspace*{\fill}\protect\raisebox{-0.5\semcm}[1.5\semcm][1\semcm]{\usebox{\lvs}}\hspace*{\fill}\protect\raisebox{-0.5\semcm}[1.5\semcm][1\semcm]{\usebox{\logo}}\hspace*{4mm}}}

%\slideframe{none}
%\slidewidth22.5cm
%\slideheight17cm
%\topsep2\semcm
% Bezeicher

\newcommand{\JaJa}{\mbox{J\&apos;aJ\&apos;a}}
\newcommand{\Tgossip}{T_{\mathrm{gossip}}}
\newcommand{\Tcompare}{T_{\mathrm{compr}}}
\newcommand{\Tseq}{T_{\mathrm{seq}}}
\newcommand{\Tpar}{T_{\mathrm{par}}}
\newcommand{\Srel}{S_{\mathrm{rel}}}
\newcommand{\Tcoll}{T_{\mathrm{coll}}}
\newcommand{\Trouting}{T_{\mathrm{rout}}}
%\newcommand{\Tpar}{T_{\mathrm{par}}}
\newcommand{\Tbroadcast}{T_{\mathrm{broadcast}}}
\newcommand{\Tcomm}{T_{\mathrm{comm}}}
\newcommand{\Tavg}{T_{\mathrm{avg}}}
\newcommand{\Tbyte}{T_{\mathrm{byte}}}
\newcommand{\Tstart}{T_{\mathrm{start}}}
\newcommand{\Tfastsort}{T_{\mathrm{start}}}
\newcommand{\Tcoeff}{T_{\mathrm{coeff}}}
\newcommand{\Tgather}{T_{\mathrm{gather}}}
\newcommand{\Tscatter}{T_{\mathrm{scatter}}}
\newcommand{\Talltoall}{T_{\mathrm{all-to-all}}}
\newcommand{\Tallgather}{T_{\mathrm{all-gather}}}
\newcommand{\Tflop}{T_{\mathrm{flop}}}
\newcommand{\Treduce}{T_{\mathrm{reduce}}}
\newcommand{\Treducescatter}{T_{\mathrm{reduce-scatter}}}
\newcommand{\Tallreduce}{T_{\mathrm{allreduce}}}
\newcommand{\lmax}{l_{\mathrm{max}}}
\newcommand{\Mall}{M_{\mathrm{all}}}
\newcommand{\Pfail}{P_{\mathrm{fail}}}
\newcommand{\Prob}[1]{\mathbf{Pr}[#1]}
\newcommand{\kopt}{k_{\mathrm{opt}}}
%\newcommand{\prob}[1]{\Prob{#1}}
\newcommand{\card}{\mathrm{card}} 
\newcommand{\co}[1]{\mathbf{#1}} % coordinate (of a cell in a CA)
\newcommand{\nb}[1]{\mathbf{#1}} % neighbor (in a CA)
\newcommand{\K}{\mathbb{K}}
\newcommand{\N}{\mathbb{N}}
\newcommand{\complex}{\mathbb{C}}
\newcommand{\Tsplit}{T_{\mathrm{split}}}
\newcommand{\Talt}{T_{\mathrm{alt}}}
\newcommand{\Tneu}{T_{\mathrm{neu}}}
\newcommand{\Tmax}{T_{\mathrm{max}}}
\newcommand{\Tsim}{T_{\mathrm{sim}}}
\newcommand{\mmax}{m_{\mathrm{max}}}
\newcommand{\work}{\mathrm{work}}

\newcommand{\splitp}{\mathrm{split}}
%\renewcommand{\Problem}{I}
\newcommand{\Proot}{P_{\mathrm{root}}}
\newcommand{\Pempty}{P_{\emptyset}}
%\newcommand{\Time}{T}
%\newcommand{\Tcoll}{T_{\mathrm{coll}}}
%\newcommand{\Trouting}{T_{\mathrm{rout}}}
%\newcommand{\Tpar}{T_{\mathrm{par}}}
%\newcommand{\Tseq}{T_{\mathrm{seq}}}
\newcommand{\Trest}{T_{\mathrm{rest}}}
\newcommand{\Tatomic}{T_{\mathrm{atomic}}}
%\newcommand{\Tsplit}{T_{\mathrm{split}}}
\newcommand{\qmax}{\bar{q}}
\newcommand{\Tpoll}{T_{\mathrm{poll}}}
\newcommand{\Ttot}{T_{\mathrm{tot}}}
\newcommand{\proc}{n}
%\newcommand{\Problem}{P}
%\newcommand{\Proot}{P_{\mathrm{root}}}
%\newcommand{\Pempty}{P_{\emptyset}}
\newcommand{\problemset}{{\cal P}}
%\newcommand{\work}{\mathrm{work}}
%\newcommand{\splitp}{\mathrm{split}}
\newcommand{\gen}{\mathrm{gen}}
\newcommand{\Time}{T}
\newcommand{\Umittel}{\bar{u}}

\newcommand{\Ht}{\tilde{H}}
\newcommand{\Tc}{T_{\rm x}}
\newcommand{\Vt}{\tilde{V}}
\newcommand{\allQ}{\breve{Q}}
\newcommand{\insertS}{{\tt insert}}
\newcommand{\insertP}{{\tt insert*}}
\newcommand{\prune}{{\tt prune}}
\newcommand{\delMin}{{\tt deleteMin}}
\newcommand{\delMinP}{{\tt deleteMin*}}
\newcommand{\deleteP}{{\tt delete*}}
\newcommand{\decreaseKey}{{\tt decreaseKey*}}
\newcommand{\delMinPTrenn}{{\tt delete\-Min*}}
\newcommand{\lesscomplex}{\leq_p}
\newcommand{\goes}{\Rightarrow}
\newcommand{\goesn}[1]{{\stackrel{{#1}}{\Rightarrow}}}
\newcommand{\goess}{\goesn{*}}
\newcommand{\uebergang}{{\,\vdash\,}}
\newcommand{\uebergangn}[1]{{\stackrel{{#1}}{\uebergang}}}
\newcommand{\ableitbar}{{\uebergang^*}}

\usepackage{relsize}

% text formatting
\newcommand{\algo}[1]{{\bf\color{KITblack70}{#1}}}
\newcommand{\term}[1]{{\bf #1}}
\newcommand{\atitle}[1]{\textsl{#1}}
\newcommand{\func}[1]{ \mathit{#1} }
\renewcommand{\var}[1]{\mathtt{#1}}
\newcommand{\aname}[1]{\textsc{#1}}
\newcommand{\software}[1]{\textsf{#1}}
%\newcommand{\code}[1]{\textsf{#1}}
\newcommand{\prb}[1]{\textsc{{\small#1}}}

\newcommand{\cov}{\operatorname{cov}}

\newcommand{\smallcaption}[1]{\caption{\scriptsize{#1}}}
\newcommand{\faint}[1]{{\color{KITblack50} #1}}
\newcommand{\etal}{et al.\xspace}

\newcommand{\nwk}{\textsf{NetworKit}\xspace}


